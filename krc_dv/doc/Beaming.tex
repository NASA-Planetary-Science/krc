\documentclass{article} 
\usepackage{underscore} % accepts  _ in text mode
\usepackage{ifpdf} % detects if processing is by pdflatex
\usepackage{/home/hkieffer/gong/tex/newcom}  % Hughs conventions
% \newcommand{\qj}{\\ \hspace*{-2.em}}      % outdent 1
\newcommand{\erfc}{\mathrm{erfc}}  % error function inside math
\newcommand{\qeq}{\hspace{25.mm}} % space original equation number to the right
%\newcommand{\qI}{\} % s
\newcommand{\bq}{$ < \! > \!   \! >$ } %  begin quote
\newcommand{\eq}{ $< \! \! < \! > $ } %  end quote
% Use only one of the following two
\newcommand{\ql}[1]{\label{eq:#1} \hspace{1cm} \mathrm{eq:#1} \end{equation}}
%\newcommand{\ql}[1]{\label{eq:#1} \end{equation} } % for final
\title{Rough Surfaces and Thermal Beaming using KRC models}
\author{Hugh H. Kieffer  \ \ File=-/krc/Doc/DV3/Beaming.tex  2016Mar}
% josh  c=602-770-9724    ??> 208-331-7998
% davidsson icarus submitted
\begin{document} %==========================================================
\maketitle
\tableofcontents
\listoffigures
%\listoftables
\hrulefill .\hrulefill
% \pagebreak

\begin{abstract}
``Thermal beaming'', the non-isotropic thermal emission of a rough surface, is
  implimented as a post-KRC process.  KRC is first run for a uniform grid of
  slopes and azimuths, azimuths being ``cases'' in a single file and each type
  52 output file being named to indicate the slope used; these KRC runs normally
  are for a planetary set of latitudes and for a planetary year (after
  spinup). An IDL program reads the KRC files and progressively narrows in on
  particular surface points and season, models of surface slope distributions,
  and viewing directions. A set of observation wavelengths can be defined, and
  the surface facets at different slopes/azimuths (and thus temperatures) are
  summed based on their abundance in the slope model and their black-body
  radiance at each wavelength, yielding a forecast infrared spectrum. While
  designed primarily for airless bodies, an ``atmosphere-present'' mode accounts
  for an atmosphere grey spectral radiance at the appropriate viewing geometry.
\end{abstract}

Remaining Steps \begin{enumerate}    % numbered items 
%\item Recover the discussion in V34 design 
%\item Define the model set needed
% \item Generate the input file(s). Models must have consistent naming
% \item Run the models
% \item Design the integation algorithm
% \item Write the integration routine
%\qi Look at starting with krchemi
% \item Run the integration, Study the results
 \item Complete the UG document

The symbols \bq and \eq  are used here to bound direct quotes from articles.

\end{enumerate}

% \hrulefill MOVE: \hrulefill

\subsection{Notation \label{nota} }
Quotes and comments on journal articles are not required to follow this notation.
Geometry notation is in \$\ref{geom}'
\\ $i$ = incidence angle on regional horizontal; or an index, hopefully clear by context
\\ $i'$ = incidence angle on individual tilt facet
\\ $\mu_0$ and $\mu_0'$: cosine of the above incidence angles
\qi  A facet is in the dark if either or both $\mu_0$'s is/are less than 0
\\ $e$ = emergence (viewing) angle on regional horizontal
\\ $e'$ = emergence angle on individual tilt facet
\\ $\mu$ and $\mu'$: cosine of the above emergence angles
\qi  A facet is invisible if either or both $\mu$'s is/are less than 0
\\ TI is thermal inertia
\\ $\bigcap$ : optional complexity

Subscripts: (not followed strictly)
\\ i : Hour, as in KRC output
\\ j : latitude,  as in KRC output
\\ J (j) : within a set of roughness mean slopes
\\ k : azimuth, as in KRC cases or tilt facets
\\ K (k): index of a KRC full tilt-set run (hundreds of slope/azimuth cases)
\\ m : slope (dip), as in KRC cases or tilt facets 
\\ l : within a set of wavelengths
\\ n : within a set of view directions 

\section{Representation}

Beaming must depend upon a roughness model and the directions of the Sun and the
viewer.

Whatever the effect at a particular geometry, its magnitude is expected to
increase more than linearly with a roughness metric, such as a mean slope,
because, by symmetry, the derivative with slope must be zero at zero slope,
suggesting that the effect may be roughly quadratic with mean slope.

If beaming increases radiation at small phase angles, there is still a problem
of defining beaming at night; whatever model is used it should be continuous
across the terminator!

If there is beaming, the surface likely has composite TI, so that spectral
assessments computed from temperature distributions along slopes are in a sense
synthetic.

For one surface point $q$, the radiance spectrum toward a viewer in direction  $\nu$ is: 

\qbn \mathcal{R}_q(\lambda,\nu) = \frac{\oint_\theta (\cos e' \geq 0) \ \epsilon_\lambda \mathcal{B}(\lambda,T_{S(s,a)}) p(s,a) \ d\theta} { \cos i} \ql{1p}
where \qi $T_S$ is the surface kinetic temperature
\qi $\lambda$ is the wavelength
\qi $\epsilon_\lambda$ is the surface spectral emissivity 
\qi $\mathcal{B}$ is the Planck function
\qi $e$ is the angle between a surface element normal and the direction to the viewer
\qi $\theta$ is an element of solid angle in the upper hemisphere above the regional horizontal defined by
\qii  slope $0 \leq s \leq 90^\circ $ relative to the regional zenith
\qii and azimuth  $0 \leq a \leq 360^\circ $ relative to some fixed direction; e.g.,'North'
\qi $p(s,a)$ is the probability of a surface element having the orientation $(s,a)$
\qi $i$ is the angle between the regional zenith and the direction to the viewer

\vspace{ 2.mm} 

For a distant view of the globe:

\qbn \mathcal{R}(\lambda,\nu) = \frac{\oint_\phi \mathcal{R}_q(\lambda,\nu)\ \ d \phi } {2 \pi} \ql{glob}
where \qi $\phi$ is an element of solid angle over the visible side of the (presumed spherical) body.
 
$T_B(\lambda) =  \mathcal{B}'(\mathcal{R}(\lambda))$ where $\mathcal{B}'$ is the inverse Planck function.

Includes the assumption that the surface roughness characteristics are the same at all azimuths.


\subsection{Fundamental modeling approaches}
Two major ways to model:
 \begin{enumerate}    % numbered items  
\item \textbf{Coupled rough surface} These incorporate radiosity, coupling
  surface elements at each time step; e.g., \qcite{Lagerros98},
  \qcite{Vasavada99}, \qcite{Rozitis11}, \qcite{Paige13} To date, these seem to
  have used a homogeneous TI.

\item \textbf{Weighted tilted models} Calculate a set of models over a range of
  slopes and azimuths, and do a weighted sum of radiances. E.g.,
  \qcite{Bandfield08}, \qcite{Bandfield09} Easy to run large set of models to
  type52. Then need reformat to file of Tsurf only. 3rd step is weighted sum
  based on roughness model and view direction. All this is similar to what was
  done for the global mean surface temperature map.
\end{enumerate}

After either of the above, could considered \textbf{Post-processing}:
Parameterize effects using angle away for Solar incidence and delta-Hour from
solar incidence.

\qcite{Vasavada99}: both flat and craters, 32x32 square grid of facets,
 include planet curvature. 

p 184.5b: The fraction of energy emitted by element $i$ that is incident on
facet $j$ 
\qbn \alpha_{ij}=\frac{1}{\pi}\frac{\cos \theta_i \ \cos \theta_j
  S_j}{d_{ij}^2} \qen 
where $\theta$ are the angles between the surface normals and the line
connecting their centers, $S_j$ is the surface area of a facet and $d_{ij}$
their separation

If $F_j$ is defined as the flux of energy leaving element $j$, then the matrix equation:
\qb F_j=A_j \left( \sum_{i=1}^N F_i \alpha_{ij} + E_j \right) \qe
$A_j$ is the facet albedo and $E_j$ is the direct insolation.

\qcite{Ingersoll92} has analytic solution for ideal spherical craters

It would be straight-forward to incorporate a spherical crater model in KRC that
used the special property of spherical Lambertian craters for the radiosity and
accurately considered the fractional sunlight and shadow for each facet, but a
model run would hold for only the specified crater depth/diameter ratio.

\subsection{Bandfield model}

\qcite{Bandfield08} used KRC for an set of slopes at 2\qd ~dip and 20\qd ~azimuth intervals; (p 143.9a)

\bq The $\theta$-bar surface model used in this work produces an array of slopes
(2\qd~ intervals) and azimuths (20\qd~ intervals) along with a weighting to
define the contribution of each slope/azimuth combination to the
measurement. This weighting is based on the Gaussian statistics of the
$\theta$-bar parameter and the projection of each surface to a plane normal to
the viewing elevation and azimuth of the measurement. Self-shadowing of surfaces
is accounted for by applying a weighting of zero to all surfaces where the
observing spacecraft is below the local horizon of the individual surface
facet. It is assumed that surfaces blocked from the view of the observing
spacecraft by other surfaces are of a random nature and do not need to be
explicitly accounted for (e.g. Hapke, 1984). \eq

Done in IDL beaming.pro @45. \\ Use $\sigma$ of 6.288, 12.74 and 19.54 to get
$\overline{\theta}$ of 5, 10, 15; see Fig. \ref{bandfieldFig2}; this does not
match Bandfield08 Fig.2 .  Using $\sigma$ of 5, 10, 15 yields
$\overline{\theta}$ of 3.98, 7.9 and 11.7

\begin{figure}[!ht] \igq{bandfieldFig2BW}
\caption[Gaussian slopes ]{Gaussian slope distributions: solid lines are to
  match $\overline{\theta}$ values shown in labels of Bandfield08 Fig.2 . Dashed
  lines have input values of 5, 10 and 15 degrees.
\label{bandfieldFig2} bandfieldFig2.png  }
\end{figure} 
% how made:  Beaming @45 455
%      1.25834      1.27410      1.30268      1.25648      1.30507      1.28221
%      1.25835      1.27410      1.30266      1.25650      1.26609      1.28221


\subsection{Treatment of roughness}  % ---------------------------------------

The individual cases in KRC do not treat statistical roughness, only allowing
the far field to be based on an RMS slope at opposing azimuth.

In summing thermal radiance from unresolved roughness elements, the view can
take into account the probabililty of facets being hidden as a function of
regional and facet emergence angles, and their relative azimuth in severla ways:
\qi Null: all facets not in title shadow are seen. 
\qi $S$: tilts are uniform in elevation 
\qi $S'$: tilt distibution is influenced by geology.

For the non-null treatment, evaluation of $S$ or $S'$ requires $\mu$, $\mu'$,
and $\psi$

\subsection{Issues}  % ---------------------------------------------
Importance of  flat far-ground? Requires KRC version 3.4+ to treat this.

 Roundoff:
\qi sumQ= summing radiance only at innermost loop 
\qi sumR =summing radiance at each loop level
\\ For 4 views and 11 wavelengths, mean value of sumQ/sumR=0.999961, StdDev=3.26951e-05

\subsection{From the literature}

\subsubsection{Emery 2014}
The Near-Earth Asteroid Thermal Model (NEATM) \qcite{Harris98} is unlikely to be
adequate for Bennau, with an TI of about 300., but it is used by \qcite{Emery14}
for their Table 5.

p24.6a, beaming parameter derived from $D_{eff}$ and $T_{SS}$

p 24.7b spherical section craters serve as macroscopic roughness elements 

$\eta$ is not used in this model. In this step, we con-
tinue to assume that the asteroid is spherical.

p32.6b I=768 for  CM meteorite Cold Bokkeveld.NWA 5515 has I=1450 

Gundlach, B., Blum, J., 2013. A new method to determine the grain size of planetary regolith. Icarus 223, 479–492.


Emery, J.P., Sprague, A.L., Witteborn, F.C., Colwell, J.E., Kozlowski, R.W.H.,
Wooden,D.H., 1998. Mercury: Thermal modeling and mid-infrared (5–12 lm)
observations. Icarus 136, 104–123.


\section{Implimentation using KRC}
Use Bandfield approach; precompute data sets of tilts as cases in
normal KRC runs; save as type 52. Azimuth at 20\qd~ intervals; slope at 2\qd~
intervals up to 40\qd; each slope is one .t52 file. 361 cases per material.

A basic simplification is to assume that all surface tilt azimuths equally probable.
 
Use equal-area zones from pole-to-poles, with one centered on the equator, so
there is an odd number total; $N=2j+1$.  Normalized area of a zone is: $\sin
\theta_2 - \sin \theta_1$ where $\theta$ is the zone boundary latitude. Thus
$\delta \equiv \Delta \sin \theta = 2/N $.  Compute the model at the
area-weighted center of the zones; $sin \theta= \delta/2+i \delta \ ; i=0:n-1$ .

 For example:
\begin{verbatim}
j=9 & n=2*j+1
dels=2./n
sa=-1.+dels*(0.5+findgen(n))
alat=!radeg*asin(sa)
 print,alat[0:n-1], format='(10f7.2)'
 -71.33 -57.36 -47.46 -39.17 -31.76 -24.90 -18.41 -12.15  -6.04   0.00
   6.04  12.15  18.41  24.90  31.76  39.17  47.46  57.36  71.33  
\end{verbatim}
\subsection{Opposing slopes}
 To account for the tendency of a slope in rough terrain to be facing slopes of
 the opposite azimuth, e.g., a west-facing slope  is likely to view
 east-facing slopes in the far field, KRC can be run with each and every
 slope and azimuth of a tilt-run using a far-field file with a tilt near the RMS
 slope value and an azimuth different by 180\qd.

This is done with the following steps: \begin{enumerate} % numbered items

\item Edit an input file similar to \nf{beamP.int} that has all the input
  parameters but ends with the last change card for the first case.

\item Edit a new KRC change file starting with a file like \nf{krc/tes/beampop} to have proper file
  names for the flat and tilt files; the slope (single input line) near the
  intended RMS slope value; and the proper names for output file for each
  azimuth.  \nf{beampop} will generate a flat file and also a type -1 for
  each azimuth (20 degree spacing) for a single slope.

\item Concatonate your \nf{.int} and \nf{beampop} files into an input file. E.g.
  \\ cat BeamZz.int beampopZz $>$ BeamopZz.inp

\item Run KRC on BeamopZz.inp, takes about 45 seconds.

\item Edit a new KRC change file starting with a file like \nf{slop.inp} to have
  proper file names for all .t52 output files (a single edit for Nslope lines,
  in emacs use !). Edit the names for all the ``far field to read'' files, a
  single edit for Nslope*Nazimuth lines; these lines must read the files
  generated in the prior step.

\np{beaming.pro} has an action @491 to generate a new \nf{slop.inp} file; first
check @11 \nv{parf} items 0:3 and @12 \nv{pari} items 1:3 

\item Concatonate your \nf{.int} and \nf{slopZz.inp} files into an input
  file. E.g.  \\ cat BeamZz.int slopZz $>$ BeamZz.inp

\item  Run  KRC on BeamZz.inp, takes about 40 minutes. This produces a .t52 file set that constitutes a ``tilt-run''  

\end{enumerate} 

\subsection{Thoughts not incorporated yet}

For distant observations, if the body shape is not spherical one could weight
each hour/latitude grid point with a form factor to reflect the relative size of
each grid facet. This is an excellent approximation (exact) where the body is
not concave

Although in the KRC runs, non-Lambertian emission can be modeled, in the
integration of rough surfaces, each facet is initially assumed to have
Lambertian emission; this Lambertian assumption could be relaxed with additional
code.
 
2016oct24 Post Banfield phone call: 
\\ Probable temperature of the far field for rough surfaces will depend strongly
on azimuth and time of day. E.g., west-facing slope in the afternoon will see
east-facing, and hence hot, slopes.
 
% \pagebreak
\section{User Guide to IDL beaming.pro}
Must first have used KRC to generate at least one full tilt-set for the object
of interest; output seasons must span one object year (IDL beaming.pro will wrap
over a gap for the last season), have a uniformly spaced set of slopes and a
complete uniformly spaced set of azimuths starting with 0 (do not include
360). See a KRC input file that does this, such as \nf{BeamA.inp} .

Notation:
\qi The ``\textbf{grid}''  is the set of hours (uniformly spaced) and latitudes (uniformly spaced in area on a sphere) in the KRC model.
\qi A ``\textbf{tilt}'' is a surface dip (uniformly spaced) and azimuth (uniformly spaced) of a surface element at a grid point.
\qi A ``\textbf{KRC tilt-set}'' is a set of KRC output files for models that all have the same input parameters except for the surface slope and azimuth. Also called a KRC  ``\textbf{run}''.
\qi A ``\textbf{view}'' specifies the direction to the viewer, initially in Hour-Latitude coordinates.

\subsection{Auto-init}
At start-up, \nf{beaming.pro} does:
\vspace{-3.mm} 
\begin{verbatim}
880.. Decomposed=0
851.. black background
860.. set to 14 colors for black background
40... View set by @15 parv
42... Prepare the wave and view vectors
45... Define the resolutions in tilt-set and theory 
256.. DEFINEKRC for current precision
\end{verbatim} 

\subsection{Step 0: optional, Plot solar declination over date}

KRC type -n files do not contain the sub-solar latitude for all seasons. 
This is an estimate based on a quadratic fit of the noon down-going insolation versus latitude.

@11, modify parts of file names: 0, 1, 2
\qi @110, 123 which does:
\vspace{-3.mm} 
\begin{verbatim}
207.. Set file names
252.. Open/Read/Close type 52 file
22... Get KRC front and hold
253.. Estimate & plot Sdec, interpolate ttt to one date
\end{verbatim}

\subsection{Step 1: Make a date file}
See \S \ref{s1} 
\\ @16, modify items 
\qi 7: Length of the target's year in Earth days. Must be same as KRC run
\qi 8: Date desired. (will be rounded to nearest integer day.) 
\qii Must be within the date range in the tilt-set files 
\\ @12,  modify items
\qi 1: Increment between slopes, degrees. Must be same as KRC run
\qi 2: Number of slopes (not including horizontal). Must be no more than KRC run 
\qi 3: Increment between azimuths, degrees. Must be same as KRC run
\\ @11, Check parts of file names: 0, 1, 2, 10, 11, 12
\\ For a \textbf{single} KRC run, do @25, then @28. Or do @111, 123
\\ For \textbf{multiple} runs: @13 to edit run-unique names, then do @141 125 [128] 123
\qii  Will prompt for file names (check and modify as needed) and float values (only item 8 is used)
\qi Output: for each tilt-run: 
\qii Date file for all slopes: parf[4]=dir +parf[5]+parf[6] e.g., BeamA5858.bin5
\qiii Tsurf [hour, latitude, azimuth, slope]
\qii Date file for flat with similar name: e.g., BeamA5858flat.bin5
\qiii Tsurf [hour, latitude]
\\ FLOW: @141
\vspace{-3.mm} 
\begin{verbatim}
     11... Modify any of strings
     16... Modify tests
/--> 133.. Set run uniq for 2nd loop
| /--> 133.. Set run uniq for 2nd loop
| |    25... Read a slope model set, interpolate to one date.
| |    28... BIN5 W t4=1season
| \<-- 1256. +++   Inner-loop increment
|     -5... Do nothing quickly
\<--- 1258. +++++ Second-loop increment
     -5... Do nothing quickly
\end{verbatim}  

\subsection{Step 2: Convert to radiance}
See \S \ref{s2}. Required input: A date file pair generated by Step 1
\qi @14: check wavelength set
\qi @11, modify parts of file names:  10,11,12,4,5
\qi @29 to read a date file
\qi @42 Prepare the wavelength array 
\qi @43 Nested 4-loop to convert Temperature to radiance
\qi @438 Write a radiance .bin5 file:
\qii File name is: parf[10]+parf[11]+parf[12]+parf[4]+'w'+parf[5]. E.g., /work1/krc/beam/BeamA5858w1.bin5
\\ For \textbf{multiple} runs: @13 to edit run-unique names, then do @142 125 [128] 123

Output:  Radiance file: [wave,azimuth,slope,hour,latitude]
\\ FLOW:
\vspace{-3.mm} 
\begin{verbatim}
     11... Modify any of strings
     14... Modify waves
/-/--> 133.. Set run uniq for 2nd loop
| |    43... Convert t4 to rad  REQ 29 42
| |    438.. Write radiance file REQ 43
| \<-- 1256. +++   Inner-loop increment
|     -5... Do nothing quickly
\<--- 1258. +++++ Second-loop increment
     -5... Do nothing quickly
\end{verbatim} 

\subsection{Step 3: Integration}
See \S \ref{s3}. Required input: one or more radiance files generated by Step 2.
\qi Check file names @11 and @13
\qi Check tilt values @12 items 1,2,3; must agree with KRC run
\qi Check wavelengths @14, a negative value ends the list
\qi If global, check views @15
\qi If local point, check views @16 items 1:6 
\\ If locale @143 \  OR \ if global @144.  Then:  @146, 125, [128,] 123

Also make an IDL save file [-.sav] with the same basic file name; these are
largely redundant but may later be handy.
 
View directions can be specified to any precision. Local view targets are
available only on grid points; interpolation of the results is probably
reasonable.
\textbf{Output}  4-D cube of radiance [waves+1,views,1+roughness, KRCrun]
\qii Last of wave dimension is the weighting applied
\qii first of roughness dimension is the smooth model
\qi  name will be  parf[10]+parf[11]+parf[8]+parf[4]+parf[9]+'w'+parf[5]  + .cub
\qiii e.g., /work1/krc/beam/BeamKLA5900H24L9w1.cub
\qii Unique part, parf[8], will be constructed as concatonation of the tilt-set names set @13.
\qii parf[4] is the date, will be the same as for the input files
\qii parf[9] depends upon the style of integration and will be constructed
\qiii  Global views output files will be 'G' 
\qiii Local views will be  e.g., H25L9 =  H + the 0-based hour index + L + the 0-based latitude index.
\\ FLOW:
\vspace{-3.mm} 
\begin{verbatim}
..====================== beaming ===================..
     11... Modify any of strings
     130.. Make list of tilt-set names
     45... Define the resolutions in tilt-set and theory 
     6.... Calc Cartesian tilt normals in S system
     207.. Set file names
     252.. Open/Read/Close type 52 file REQ 207
     62... Set to a single grid point and hour traverse
     622.. Add N/S transect AFTER 403
     42... Prepare the wave and view vectors
     1461. Create storage
/--> 133.. Set run uniq for 2nd loop
|     439.. Read radiance file
| /--> 134.. Increment theta-bar
| |    46... Compute distribution for KRC slopes REQ 29 or 439, 45
| |    64... Integrate roughness REQ 42 and  46 or 439 
| |    1434. Save into n+2 D arrays
| |    1161. +1 Optional stop after one set
| \<-- 1256. +++   Inner-loop increment
|     1162. +2 Optional stop after inner loop
\<--- 1258. +++++ Second-loop increment
     1164. +4 Optional stop after second loop
     1435. SAVE (make and store 4-D cube)
\end{verbatim}

 Then can do @7 to read the cube, which will yield a plot for each tilt-set
\qi and @71, which will plot traverses for all tilt-runs
\subsection{Analysis}
@7 Read a cube
\qi @71  Plot for each tilt-set
\qi @72 Traverse plot for all runs, for the roughest model


% \pagebreak
\section{Geometry notation \label{geom}}

 The notation system used here was developed so that variable names in code can
 follow closely the mathmatical representation.  [ Largely extracted from \nf{-/xtex/Geomath/matrix.tex} ]  
\\ Briefly, in code names are [to][from][system][component] with component ``xx'' representing an X,Y,Z triple (``xxx'' indicates that this is an array of such triples)
\qi A final ``u'' indicates this is a unit vector. 
\qi In code, generally use lower-case version of the upper-case math symbol
\qi E.g., vqdxu is $\qf{VQ_D}$ unit vector from $Q$ to $V$ expressed in the $D$ system.

\vspace{2.mm}
Locations (vector ends) used here:
\\ N = Unit vector along the local surface normal.
\\ P = center (of mass?) of the target body (Planet or satellite)
\\ Q = surface intercept location (of the instrument optic axis)
\\ V = Vehicle or spacecraft; e.g., viewer, imager, camera. Where one is looking FROM
\\ W = local slope element of a KRC model
\\ Z = generic, the +Z-axis direction of the given coordinate system or surface normal.
 
\vspace{2.mm}
Orientation systems used here: all are X,Y,Z right-hand. 
\\ A = Astronomic: Master reference inertial system (ICRF or J2000)
\qi  +Z toward the Earth's north pole, +X toward the vernal equinox.
\qii +Y is $Z \times  X $, the third axis is defined by a cross product in all systems. 
\\ D = Day:        Target body spin axis and true solar midnight
\qi +Z toward body's right-hand spin axis, +X in the true solar midnight meridian.
\qii Hour increases in a right-hand sense
\\ S = Surface:    Regional 'horizontal' surface parallel to the ellipsoid surface
\qi  +Z toward the regional zenith, +Y toward north (degenerate at a pole).
\qii  +X right-handed (East], = Y cross Z. X and Y are in the ``horizontal'' plane.

\vspace{2.mm}
Other symbols:
\qi $\bullet$ is the dot-product 
\qi $\times$ is the cross-product 
\qi $\overline{QV}$ is the magnitude (distance) between $Q$ and $V$
\qi $\star$ indicates conventional matrix rotation, implemented in IDL by the \#
 operator.
\qi \trm{DA} is the rotation matrix taking vectors from the $A$ to the $D$ system
\qii Last two letters in code are 'rm'. May to append a 9 if a 9-element 1-D array rather than a 3x3 array.
\qi  $\qct{DA}$ is the coordinate transform that rotates the axes of the $A$ to 
coincide with the axes of the $D$ system
\qii  The last 2 letters in code are 'ct'
\qi E.g., $\qf{ZQ_S}$ is out along the ellipsoid normal at (latitude,Hour) in the $S$ system;  components are by definition [0,0,1] 


Note that $\qrm{AB}^\mathrm{T} = \qrm{BA}  = \qct{AB} $
where $^\mathrm{T}$ means the transpose.

BEWARE: It is a peculiarity of my IDL rotation package that vector rotations to
S from D are done by the ROTVEC routine using the \tct{SD} matrix constructed by
use of the ROTAX routine to rotate the axes of the D system onto the axes of the
S system.


 % \pagebreak
\subsection{Specifics \label{geos} }
KRC azimuths are: Increasing east from N (clockwise) [but be aware of sign
  error that had them West from North in versions 232 (and possibly earlier) to
  342].

 Specify viewer directions and surface points in the $D$ system, but do inner
 loop calculations in the $S$ system for efficiency. Generate slope normals in
 the S system $\qf{ZQ_{j,k}}$ where $j$ is the azimuth index and $k$ is the
 slope index.
\qi Get $\qf{VP}  $ direction in D system
\qii Assume for now the target body is small relative to the observer distance so that $\qf{VQ} \equiv \qf{VP}$ 
\qi Then the view factor for a tilt element is  $f_{jk}=\qf{VQ} \bullet \qf{ZQ_{j,k}} $ and must constrain  $f \geq 0$.
\qi $\qf{VQ_S} = \qrm{SD} \star  \qf{VQ_D}$  

Make \tct{SD} coordinate transform by rotating the D axes to coincide with S axes
\qi 0) Start with identity matrix: 
\qi 1) Rotate around the pole from midnight to Q hour \qt Rotate around Z by 15*Hour of S
\qi 2) Rotate around Y from N pole to Q latitude \qt Rotate around Y by (90 - Lat. of S). X is now toward South
\qi 3) Rotate around Z 90\qd~ to move +Y from east to north \qt . Or exchange axes

\section{Intermission on definition of mean value}

Given a continuous population function $p(x)$ evaluated at a discrete set of points $x_i$ to generate the set $p_i$.; e.g., $x$ could be angle or $\tan \theta$ .
\qi the mean value of $p$ is $ m=\frac{1}{N} \sum_{i=1}^N p_i $ and
 the root-mean-square (RMS) of $p$ is $ r=\sqrt{ \sum_{i=1}^N p_i^2 \big/ N }$.
\\ Assuming that the $x_i$ are uniformly distributed, then
\qi the mean value of $x$ is $ m_x=\sum_{i=1}^N p_i x_i \big/ \sum_{i=1}^N p_i $ 
and RMS of $x$ is 
$ r_x=\sqrt{ \sum_{i=1}^N p_i  x^2_i  \big/ \sum_{i=1}^N p_i }$.

If the distribution is normalized so that $\sum_{i=1}^N p_i =1$ 
 then  $ m_x=\sum_{i=1}^N p_i x_i $
 and the RMS is $ r_x=\sqrt{ \sum_{i=1}^N p_i x^2_i }$

\vspace{4.mm}

If the intervals represented by the distribution are not uniform but have width $w_i$, 
\qii then $ m_w=\sum_{i=1}^N w_ip_ix_i \ \big/ \ \sum_{i=1}^N w_ip_i$ and 
\qb r_w=\sqrt{ \sum_{i=1}^N w_ip_i x^2_i\ \Big/ \  \sum_{i=1}^N w_ip_i } \qe


Specifically, if computing the $p(s)$ distibution at uniform points in angle,
but doing the statistics along a scale of slope, then $w= \Delta_i (tan \theta)
\sim d \ \tan \theta = \sec ^2 \theta \ d \theta $, where $\Delta_i$ is assessed
between the midpoints $(\theta_{i-1}+ \theta_i )/2 $ and $(\theta_i +
\theta_{i+1} )/2 $. Note that scaling $w$ or/and $p$ makes no difference.

\subsection{Just what is the mean slope}
Options in beaming.pro @451
\qi Can try either 2 or pi as divisor in exp
\qi Can try either Hapke or RMS for a roughness index
\qi ? can weight with $d \theta$ or  $d \tan \theta$


\section{Algorithm} 

Initially, assume viewer is far away, so use orthographic
projection. Perspective maps are geometrically much more complex and are not
needed unless a spacecraft is within about 20 body radii.

.
\\ All processing is done by IDL \np{beaming.pro}. The ``@'' symbol is an address in a large case statement.

\subsection{Processing step 1; one date \label{s1}} 
Interpolate a KRC tilt-set to a single season: reformat to file of
Tsurf[hour,lat,season,azimuth,slope].  Output 1.3Mb +7.8 kB (flat)
.
\\ Make a date file:
\qi Specify the KRC tilt-set
\qi Specify the date
\qii Loop on files (slopes) in the model set  @25
\qii Read the .t52 file, one slope of [hour,item,latitude,season,case=azimuth]
\qiii Interpolate Tsurf in date;
\qiii Store in $T_S$ array with same order: [hour,latitude,azimuth,slope] 
\qii Write as .bin5 named with the date  @28


\vspace{3mm}
Read all the KRC files and combine at one surface location
\qii First file, single flat case: extract latitude, dates and set some sizes 
\qii Second file, with many cases: extract azimuths and set that size 
\qi Linear iterpolation in date, make  t4= Tsur[hour,lat,case=azi, slope] 
\qi  and: tflat =Tsur[hour,latitude]
\qiii If also using Tatm, will need another dimension (but do not call it t5)
\qi Save as two bin5 files named to include date, 
\qii header contains: latitudes, azimuths, slopes and the single date

\subsection{Processing step 2; convert to radiance \label{s2}} 
Integration must be done in radiance, so convert Brightness-temperatures to radiances
\\ Specify the effective wavelengths [firm code, or revise @14]
\\ Read the date file: @43
\qii If appropriate, reduce hour density to: number of hours $\sim$ 2* Number-latitudes 
\qi Convert to Tsurf to radiances and reorder: [wave, azimuth, slope, hour, latitude] 

% \pagebreak

\subsection{Processing step 3; Integration \label{s3}} 
\vspace{3.mm} Impliment \qr{glob} as, only where $\cos e'_n > 0 $: 
\qb \mathcal{R}(\lambda)_n = \left[ \sum_{j=1}^\mathrm{nLat}
  \sum_{i=1}^\mathrm{nHour} \ \sum_{m=1}^\mathrm{nSlope}
  \sum_{k=1}^\mathrm{nAzi} p(\theta_m,\phi_k)\cos e_n' \ \epsilon_\lambda \left[
    \left(1-S(i,e,\psi,\theta) \right) \mathcal{R}(\lambda,k,m,i,j)
    +S(i,e,\psi,\theta) \mathcal{R_S}(\lambda,k,m,i,j) \right] \right] \qe

\qbn / \ \left[  \sum_{j=1}^\mathrm{nLat} \sum_{i=1}^\mathrm{nHour} \sum_{m=1}^\mathrm{nSlope} \sum_{k=1}^\mathrm{nAzi} p(\theta_m,\phi_k) \cos e'_n \right] \ql{igl}

where the subscript $n$ runs over a set of viewer
directions. $S(i,e,\psi,\theta))$ is the shadow function being used and
$\mathcal{R_S}$ is the radiance for the temperature in the shade.

Basic shadow function: use \qcite{Smith67} for the fraction of surface NOT in shadow; using his notation. % (terms reorganized)

Paraphrazing: the probability that a point on a random rough (Gaussian height
distribution) surface, will not lie in shadow when the surface is illuminated
with a parallel beam of radiation at a fixed angle of incidence to the mean
plane.

\qbn \underbrace{S(i,\overline{\theta})}_{\mathrm{here}} \Leftarrow  \underbrace{ S(\theta)}_{\mathrm{Smith67}} = \frac{  1- \frac{1}{2} \erfc \left(   \mu / \sqrt{2} w \right) } 
{\left( \sqrt{2/\pi} \cdot \frac{w}{\mu}
e^{-\mu^2 / 2w^2} -\erfc \left(  \mu / \sqrt{2} w \right) \right)/2  +1 } \qeq (24 \ \mathrm{and} \ 21) \qen

where $i$ is the incidence angle,  $\mu = \cot i $, $w^2$, is the mean square surface slope, $w \equiv \overline{\theta} $


\subsubsection{RMS dips and slopes}
For a surface with 
\qb P(\tan \theta)=\frac{\tan \theta}{\tan^2 \theta_0} \cdot e^{-\frac{\tan^2 \theta}{2 \tan^2 \theta_0} } \leftrightarrow  P(s)=\frac{s}{s_0^2}e^{-(s^2 / 2s_0^2)} \qe ;
from \qcite{Shepard95} Eq. 1

Using $s\equiv \tan \theta$, the mean-square slope is  
\qbn w^2= \int _0^\infty P^2(s) \ ds \Rightarrow
  \int _0^\infty \left( (\frac{s}{s_0^2} \ e^{-\frac{s^2}{2 s_0^2} } \right) ^2 \ ds
 =\frac{1}{s_0^4} \int_0^\infty  s^2 e^{-s^4/a} \ ds 
 \Rightarrow \frac{1}{s_0^4}  \bigg[_0^\infty  - \frac{x^3 \Gamma \left( \frac{3}{4},\frac{x^4}{a} \right) }{ 4 \left( \frac{x^4}{a} \right) ^{3/4} }   \qen
where $\Gamma(a,x) $ is the incomplete Gamma function



\vspace{-3.mm} 
\begin{verbatim}
 Note: use http://www.wolframalpha.com/ for integral
\end{verbatim}  
% Testing symbol sizes: \qb a / b] \ \ a \big/ b \big] \ \ a \Big/ b \Big] \ \  a \bigg/ b \bigg]  \ \  a \Bigg/ b \Bigg] \qe

\subsubsection{Modification for dip} %.........................................

Allow modification of shadowing to account for a geologic influence on the
distribution of dip with elevation; see the section on ``Implication of
geology'' in \nf{slopes.tex}. A simple approach is $
S'(i,\overline{\theta},\delta) = S(i,\overline{\theta}) F(\delta)$ where
$\delta$ is the dip angle of a facet, and require $\int_0^{\pi/2}
F(\delta)p(\delta) =1$ where $p(\delta)$ is the population function of dips and
$\int_0^{\pi/2} p(\delta) \ d\delta =1$.

Define $F$ as a Chebyshev polynomial of the first kind of degree 2 in
$x=2\delta/\delta_\mathrm{max} -1$ [which ranges over -1:1]; i.e.,
$Y=1+c_1x+c_2(2x^2-1)$. Define $F=Y/\int_0^{\pi/2} Y(x)p(x)\ dx $ to accomplish
the required normalization.

However, this approach could cause $S'$ to fall outside the physical limits: $ 0
\leq S' \leq 1$

Physical requirements force some complication; i.e. that $F$, and hence $Y$,
must not be negative anywhere, that $ 0 \leq S' \leq 1$, and that $S'
\rightarrow 0$ as $i \rightarrow 90$\qd~ and $S' \rightarrow 1$ as $i
\rightarrow 0$.

 The $S$ function has all attributes required of $S'$, which suggests adjusting
 $\overline{\theta}$ as a function of $\delta$; i.e., \qbn
 S'(i,\overline{\theta},\delta) = S(i,F(x)\overline{\theta},\delta) \qen Then,
 $Y$ decreasing with $\delta$ would decrease the shadowing of the steeper
 slopes; e.g., $|c_2| << |c_1| $ and $c_1 < 0$ .

MORE?

\subsubsection{Azimuth relation}  %.......................................

For relative azimuth dependence, allow either the Hapke relation $f(\Psi)=\exp
  \left( -2 \tan \frac{\Psi}{2} \right)$ or the simple  $f(\Psi)=\frac{1+ \cos \psi}{2}$.

Temperatures in shade are set to the minimum diurnal temperature for each facet, which is an extreme.

 See \S \ref{geom} and \S \ref{nota} for definitions and notation

 \pagebreak 
\subsubsection{Actions and loops}  %..............................
.
\\ 130.. Make list of tilt-set names
\\ 45... Define the resolutions in tilt-set and theory
\\ 6.... Calc Cartesian tilt facet normals in S system
\qii Compute normals for each tilt (slope, azimuth) in the S coordinate system: $\qf{NW_S}$
\\ 207.. Set file names
\\ 252.. Open/Read/Close type a 52 file to define the hour/lat grid
\qii  Define the surface latitude and hour grid (based on KRC file)
\\ ------ For a local target 
\qii Specify sub-viewer directions as hour and latitude; i.e, in the D system
\\ 403.. View is symmetric scan in hours
\\ 404.. Add N/S transect AFTER 403
\\ 42... Prepare the wave and view vectors
\qii Compute vectors to viewer in D coordinates: $\qf{VQ_D}$
\\ 62... Set to a single hour and lat grid point
\\ ------------------ local \ / \ global
\\ ??
\\ 63... Set to global hour and lat grid
\\ ------------------ end global
\\ 1461. Create storage
\\ $>>$ Top of Loop for each KRC run
\qi  133.. Set run uniq for 2nd loop
\qi 439.. Read date-radiance file
\qi $>>$ Top of Loop for each $\overline{\theta}$
\qii 134.. Increment $\overline{\theta}$
\qii  46... Compute the fractional abundance of KRC slope interval
\qii  64... Integrate roughness REQ 42 rrr 46 
\qiii \ddag Set any shadow or azimuth flags
\qiii \ddag Compute shadow slope factors
\qiii \ddag Find the minimum temperature for each case, convert to radiances 
\\ .  \hrulefill 5-deep Loops within @64 \hrulefill \hspace{3.in}
\\ For each grid point: latitude, then hour 
\qi Compute zenith for grid-point in the D coordinate system: $\qf{ZQ_D}$
\qi Ensure it is visible in at least one view, if so: for each view
\qii View factor is $ \cos i =\qf{VQ_D} \bullet \qf{ZQ_D}$ . Require at least one $> \epsilon = $ 1.\xEm6
\qi Compute rotation matrix to S from D, \trm{SD}.; see \S \ref{geos}  
\qii \ddag  $\bigcap$ Compute relative azimuth
\qii \ddag $\bigcap$ Compute shadow probability, possibly for each dip
\qi Extract radiances for all azimuths and slopes for this grid point
\qiii Compute probability $H(i, \overline{\theta}, d)$ of projected shade; Set radiance as $\mathcal{R}' = (1-H)\mathcal{R} + H \mathcal{R_\mathrm{min}}$ for each $\lambda $
\qi Rotate viewer positions into S system, $\qf{VQ_S} = \qrm{SD} \star  \qf{VQ_D}$
\qi Compute the view-factor for horizontal (flat) surface and sum the smooth radiance
\qi For each tilt: slope, then azimuth
\qii For each view position
\qiii Compute the view-factor (cosine of emergence angle) $\qf{VQ_S} \bullet \qf{NW_S}$. 
\qiiii \ddag $\bigcap$ $H(e,\overline{\theta}, d) =1-S' $ of projected obscuration; apply/replace ??  this factor to view factor .
\qiii Add the radiance vector (a set of wavelengths) weighted by view-factor $ \geq 0$
\qiii Sum the view factors
\\ Convert average radiance (radiance sum / view-factor sum) for each view to Tb. Save as [wavelength, view, smooth\&rough]
\\ .  \hrulefill end of @64 \hrulefill \hspace{3.in}
\qii 1434. Save into n+2 D arrays
\qii . [1161. Optional stop after one set]
\qi $<<$ 1256. Increment $\overline{\theta}$ until done
\qi . [1162. Optional stop after inner loop]
\\ $<<$ 1258. Increment KRC run until done
\\ . [1164. Optional stop after second loop]
\\ 1435. SAVE. Store 4-D augmented cube file; [waves+1, views, 1+roughness, krcRun]

\vspace{3mm} Projected obscuration; The dip factor addresses the geologic bias on 
the probability of emergence obscuration, but no basis has been developed for assigning a temperature to the surface causing the hiding; a neutral approach is to simple not count the contribution of that surface to radiance or area summation.

\vspace{3mm} NOT YET: Makes sense to use a single atm temperature. Use the Tatm
of all models weighted with the surface abundance based on slope distribution,
not the view factors.


\subsection{Analysis}

 if local do @62, if global do @63, then 7,... 71

\subsubsection{Narrowing of data size, and file naming}
A typical set of KRC type 52 files generated for beaming is 774 Mb [Ts
  only]. Constructing a set of Tsurf files for one date, @25, yields
t4=[hour,[Ts/Ta],lat,case=azimuth, slope] and a file for the flat case, total
1.32Mb. 
 
\begin{verbatim}
OBSOLETE ??

first run of 4 views without cos i test Elapsed time1=        2.4648681
  with  1.28600

at !dbug=2 can see that model temperatures have some irregularity.
QLAT=-6.04000 QHOUR=4.50000  mostly about 0.3K , at max slope, max of 1 k
\end{verbatim}

.
\\ BeamXii.t52 \ Tilt runs: where X is a run letter and ii is the slope in degrees
\qi Contains a case for each azimuth
\\ BeamXnnnn.bin5 \ Date files: where nnnn is the modified julian day
\qi 48 19 18 20  [hour,latitude,case=azimuth,file=slope]  Real*4
\\ BeamXnnnnflat.bin5  \ Date file for horizontal surface:
\qi 48 19 Ts[hour,latitude]   Real*8
\\ BeamXnnnnQii .bin5  \ Ts for a single point (latitude and Hour) at a specific date.
\qi 18 21 Ts[azimuth,[flat,slope]]
\\ BeamK5858w1.bin5 radiance file specific run=K, specific date=5858 wavelength set 1
\qi 11 18 21 48 19 [wavelength,hour,latitude,case=azimuth,file=slope]

\vspace{2mm} .
\\ KLA5858H25P9w1.cub   4-D radiance cube
\qi 12 21 4 3  [waves+1,views,1+roughness, krcRun]
\qiii Header contains wavelengths and views
\qii Last waves is the weighting divisor
\qii First roughness is the smooth (all flat) model
\\ KLA5858H25P9w1.sav  /  same as above .cub:  @1325
\qi saving:  avrr4,weir4,avrs4,weis4,vuin,sro,wavet
\\ KLA5858w1.cub   Early radiance cube
\qi 12 4 4 3
\\ KLA5858w1.sav 
\qi same as above, saving:  avrr4,weir4,avrs4,weis4,vuin,sro,wavet



\subsubsection{Sdec}
Solar declination is not in the type 52 file. Can estimate it for each season by
fitting the down-going Visible radiation at noon as a function of latitude, and
finding the maximum. Typical result is shown in Figure \ref{beam251}
\begin{figure}[!ht] \igq{beam251}
\caption[Estimated sub-Solar Latitude]{Sub-solar latitude derived from the
  down-going visible radiation as a function of latitude.
\label{beam251}  beam251.png }
\end{figure} 
% how made: beaming @251


\subsection{Thoughts}

 Small-scale effects (less than about 5 times the diurnal skin depth) can only
 dimish the thermal differences with slope and hence the beaming effect.

Surface hiding at large emission angles must attenuate the shallow slopes. 
Shadowing at large incidence angles must attenuate the shallow slopes.

\section{Bennu and OSIRIS-Rex}  % ---------------------------------------------

\qcite{Emery14}
\qi bulk thermal inertia of 310 $\pm$ 70 \quti
\qi use visible geometric albedo  0.046  $\pm$0.005
\qii 23.2a: $p_v = ( D_o/ D_{eff})^2 10^{-0.4H_v}$  \ \ $D_o$ taken as 1329 km \ \ $H_v$ =20.51 $\pm$0.10
\qii $ D_{eff}$ about 600 m; Table 5
\qii yields $p_v=$ 0.0306723
\qi The Bond albedo is calculated as $A_B = p_v q$ 
\qii p 23.0b,  $q$ is the phase integral (0.367 ± 0.045),
\qii yields $A_B =$  0.0112568
\\ .
\\ \qcite{Chesley14}
\qi bulk density at 1260 $\pm$ 70 kg/m$^3$ and macroporosity 40 $\pm$ 10\%.

Bennu year is 436.42 days.  Beam runs are 41 seasons spaced by 10.9106 days,
5764.2720 to 6200.6960; 

\vspace{2.mm} OSIRIS-Rex: Arrival is 2018 august, [aug03 is MJD 6789, handy],
Survey begins 2018Oct, Return window opens 2021march, Earth landing is
2023sept24. Thus, spacecraft is near Bennu for a little more than 2 of its
years. MJD of the first three events is: 6787, 6848, 7730.
\qi Equivalent dates:  5914.15, 5975.15 and 5984.31, respectively.
\qi Solar declination: 33 N, 22 N, 18 N 
\\ In the KRC runs, Extreme seasons are:
\vspace{-3.mm} 
\begin{verbatim}
Ls=0:    index=39, djmm=6189.8, sdec= +0.65    runset 6190
   N:    index=12, djmm=5895.2, sdec= 33.16    runset 5900
Ls=180:  index=24, djmm=6026.1, sdec= -2.79 
     S:  index=30, djmm=6091.6, sdec=-33.07    runset 6092
\end{verbatim} 

2016oct12, Looked through \qcite{} and used orbital and rotatioinal parmeters to
make an additional entry in NEO.tab; also found Bond Albedo = 0.0173 +/-.0027
(1-sigma).

\subsubsection{Pole}

In -/porb/small.tab, as the 6'th object: 
\vspace{-3.mm} 
\begin{verbatim}
101955 Bennu 2455562.5   /6   from Hergenrother 2014
e       0.2037451146135014      2.123081E-08    J2000 orbital values 
a 	1.126391025571644       4.111300E-11	AU    source cited as 
q 	0.8968943569669300      2.390158E-08 	AU    JPL IOM 343R-13-00
i 	6.03493867976381        4.721372E-08	deg   Chesley et al 2014
node 	2.06086819910204        6.531279E-08	deg    
peri    66.22306886293201       9.685025E-08	deg
M 	101.70394725713800      2.6E-06 	deg  P=436.648727924 days
rot_per 4.297461   ± 0.002 h, retrog. Nolan et al 2013.  JPL has  4.288 h 
PoleRA  86.64   Nolan 2013 : β = −88° (lat), λ = 45° and an uncertainty of 5°.  
PoleDec -65.11    in ecliptic coordinates,  Convert using eclip2equat.pro 
merid   0.         Obliq=176 +/- 2
\end{verbatim}

% after lon/lat switch correction, equatorial pole values agree with conversion using NED: NASA/IPAC EXTRAGALACTIC DATABASE, to $,.0001$ degree.

\subsection{Production Runs}  % ----------

\begin{table} [!h]
\caption{KRC tilt-set runs. All have: EMISS=1., RLAY=1.15, FLAY=.12, N1=37.; for which the bottom is at  D=121.7 or 1.5048m }
\label{runt}
\begin{center}
\begin{tabular}{| c  r  r  c | c | r | } \hline \hline
Label & Albedo & Inertia & Photom. & Other conditions & run time\\
       &       &        & Function &  & seconds\\  \hline
K & .03 & 100 & Kheim & & 2474 \\ 
L & .03 & 100 & Lambert & & 2386 \\   % 1592+794
A & .10 & 100 & Kheim & &  2373 \\ 
I & .03 & 200 & Kheim & &  2490 \\ 
J & .03 & 400 & Kheim & & 2480 \\ 
D & .03 & 100 & Kheim & I=600 below D=.3 \\
P & .03 & 100 & Kheim & far-field is slope=30 at azimuth+180 & 2458\\ 
 \hline
\end{tabular} \end{center}
\end{table}

 Start with basic model representative of asteroids using a nominal orbit and
 spin axis (e.g., Bennu). Offset individually various model parameters to
 generate partial derivatives.

 All runs have a flat model used for the far field and a set of slopes spaced a
 2 degress up to 60 \qd, with 18 azimuths spaced by 20\qd. Runs have 40 seasons
 witha 3-yr spinup and use 19 latitudes centered in equal-area zones. Runs use
 N1= 37 layers with the first physical layer having a thickness of FLAY= 0.12
 the diurnal skin depth (D) and the thickness ratio of successive layers being
 RLAY= 1.15 , Using number of output hours N24= 48 and number of latitudes N4=
 19 yields an hour density 1.26 that of latitude, so areas in the hour-latitude
 grid are close to equant.  Input parameters are largely Version 3.3 default
 values except as listed in Table \ref{runt}. Output 30 files of 38.6Mb each and
 one flat of 2.1Mb for a total of 1.16GB for one tilt-set. One run takes about
 40 minutes (0.77GB and 27 minutes if go only to 40 \qd~ slope).

Tilt-set runs to date have files for each slope labeled BeamXNN.t52 where X is
are run Label (see Table \ref{runt}) and XX is the slope, XX=00 is the flat
file.

Through an error in the direction of the spin axis, the computed sub-solar
latitude ranged over $\pm 33$\qd, a factor of 6.72 larger than for Bennu, for
which on KRC seasonal dates the range is $\pm 4.93$\qd; See
Fig. \ref{BenSdec}. This error was discovered after 5 tilt-set runs, and
retained for the rest of the runs as potentialy allowing more informative
analysis.
\begin{figure}[!ht] \igq{BenSdec}
\caption[Sub-solar latitude]{Subsolar latiitude (solar declination) for Bennu as
  a function of date; the values are derived from a quadratic fit to the
  down-going insolation at noon for each KRC season.  Solid lines and + signs
  are for Bennu; dashed lines are value for the beaming runs, reduced by a
  factor of 6.72 and offset by 20 days
\label{BenSdec}  BenSdec.png  }
\end{figure} 
% how made: 

% \pagebreak
\bibliography{heat,moon,mars}   %>>>> bibliography data
\bibliographystyle{plain}   % alpha  abbrev 

\appendix %====================================================================

\section{Parameters}
Leading '-' indicates value that must agree with a KRC model set. A leading '+' indicates a value that is required for the minimum full run.
\vspace{-3.mm} 
\begin{verbatim}
@11 parf: File names
+  0 DIR for krc file                    = /work1/krc/
+  1  stem " "                           = BeamK
  2  " slope part [.t52]                 = 00
  3 stem for 2nd set                     = BeamBen
  4 Output dir                           = /work1/krc/beam/
  5 " stem                               = BeamK
  6 mjd [auto][.bin5]                    = 5858
  7 waveset number                       = 1
  8 Surface site name, start with letter = Q1
  9 CubUniq                              = KLA
 10 Surface point                        = H25L9

@12 pari: Integer values
       0       0  set komit
 -     1       2  slope increment, degrees
 -     2      20  Number of slopes
 -     3      20  Azimuth increment, degrees
 -     4      48  Number of hours output
       5       6  STOP after: +1=seq +2=loop1 +4=loop2
       6       7  KRCCOMLAB:+1=real +2=int +4=log
       7     -77  spare
       8     -77  spare
       9     -77  spare
      10      25  Hour index ->ihour, set to ihot
      11       9  single Lat index ->jlat, set to jeq
      12      31  Index of season near Ls=0

@13 Tilt sets        Number adjustable
   0   K
   1   L
   2   A

@14 wavin: Wavelengths          Number adjustable
+      0      3.00000  \  Wavelengths
       1      5.00000   | in micrometers.
       2      8.00000   | First non-positive
       3      11.0000   | value terminates
       4      15.0000   | the list.
       5      20.0000   |
       6      25.0000   |
       7      50.0000   |
       8      100.000   |
       9     -1.00000  /

@15 parv: Views                  
+      0      12.0000  Hour  1\  Sub-viewer
+      1      0.00000   Lat. 1 | Latitude
       2      3.00000  Hour  2 | and hour
       3      0.00000   Lat. 2 | in pairs.
       4      5.00000  Hour  3 | First negative
       5      89.0000   Lat. 3 | Hour terminates
       6      13.0000  Hour  4 | the list.
       7      90.0000   Lat. 4 | 
       8     -1.00000  Hour  5 | 
       9      0.00000   Lat. 5 | 
      10     -1.00000  Hour  6 | 
      11      0.00000   Lat. 6/

@16 parr: Float values 
       0     0.100000  del Degree in model @45
       1      13.0000  View E/W central hour
       2      15.0000   " " del deg @401
       3      0.00000   " " Number to each side  INT
       4      5.00000   "  N/S central lat
       5      15.0000   " " del deg @402
       6      5.00000   " " Number to each side  INT
-      7      436.423  Length of a year
+      8      5858.00  Target MJD
+      9      5.00000  theta-bar, deg
+     10      10.0000   " " delta
+     11      33.0000   " " maximum
\end{verbatim}  

\section{Sequences for Steps}
\vspace{-3.mm} 
\begin{verbatim}
At start:.
850, 850, 861    Set colors
42... Prepare the wave and view vectors
45... Prepare for slope distrib.
256.. DEFINEKRC for current precision REQ 22 

@111 is
207.. set file names
25... Read a slope model set
22... Get KRC front and hold   23: Print krccom
27... Clot [hour,lat] season 20, solazi=0
-1... pause
272.. Clot [hour,case] season 20, lat=0
-1... pause
28... BIN5 W t4=1season

                  Process beaming runs to date files:

Step 1: Generate a date file
16... Modify file names
25... Read a slope model set, interpolate to one date.
28... BIN5 W t4=1season

Step 2: Process date files to fluxes
11... Modify file names
29... BIN5 R t4 = 1 season
43... Convert t4 to rad  REQ 29 42
438.. Write radiance file REQ 43

View Traverses for one point
     11... Modify any of strings
     130.. Make list of tilt-set  names
local
     45... Define the resolutions in tilt-set and theory
     6.... Calc Cartesian tilt normals in S system
     207.. set file names
     252.. Open/Read/Close type 52 file
     403.. view from pari[7] hours to each side
     404.. add N/S transect AFTER 403
     42... Prepare the wave and view vectors
     62... set to a single hour and lat grid point
Global
     401.. View set by @15 parv
     42... Prepare the wave and view vectors
     45... Define the resolutions in tilt-set and theory 
     6.... Calc Cartesian tilt normals in S system
     207.. Set file names
     252.. Open/Read/Close type 52 file REQ 207
     63... Set to global hour and lat grid
both
     1461. Create storage
/--> 1313. Start of each inner loop
|     439.. Read radiance file
| /--> 1323. increment theta-bar
| |    46... Compute distribution for KRC slopes REQ 29 or 439, 45
| |    64... Integrate roughness REQ 42 rrr 46 
| |    1324. save into n+2 D arrays
| |    1161. +1 Optional stop after one set
| \<-- 1256. +++   Inner-loop increment
|     1162. +2 Optional stop after inner loop
\<--- 1258. +++++ Second-loop increment
     1164. +4 Optional stop after second loop
     1325. SAVE (make and store 4-D cube)

 Global view:
     130.. Make list of tilt-set names
     45... Prepare tilt distribution
     6.... Calc Cartesian tilt normals in S system
     207.. Set file names
     252.. Open/Read/Close type 52 file
     62... Set to a single hour and lat grid point
     63... Set to global hour and lat grid
     1341. Create storage
/--> 1313. Start of each inner loop
|     439.. Read radiance file
| /--> 1323. Increment theta-bar
| |    46... Compute distribution for KRC slopes REQ 29 or 439, 45
| |    64... Integrate roughness REQ 42 rrr 46 
| |    1324. Save into n+2 D arrays
| |    1161. +1 Optional stop after one set
| \<-- 1256. +++   Inner-loop increment
|     1162. +2 Optional stop after inner loop
\<--- 1258. +++++ Second-loop increment
     1164. +4 Optional stop after second loop
     1325. SAVE (make and store 4-D cube)
\end{verbatim}
\vspace{-3.mm} 

%\clearpage %__________________________________________________________

\section{Testing sep 8}
\vspace{-3.mm} 
\begin{verbatim}
       9      15.0000  theta-bar, deg
normal, store zz0
Tcon    store zz1
beaming Enter selection: 99=help 0=stop 123=auto> 69
Radiance out/in: mean. StDev=     0.123866     0.000622767

theta-bar=2 store zz2  Radiance out/in: mean. StDev=     0.123866   0.000622767
\end{verbatim} 
After scaling, radiance spectrum is shown in 
\ref{beamKa}
\begin{figure}[!ht] \igq{beamKa}
\caption[First try]{Brightness temperature spectra after emperical scaling.
\label{beamKa} beamKa.png  }
\end{figure} 
% how made: 


run with theta=2, fpp=  0.129009     0.558479     0.271261    0.0392474   0.00196731  3.55481e-05  2.34593e-07 ...

after scaling: Fig.
\ref{beamK2}
\begin{figure}[!ht] \igq{beamK2}
\caption[nill slope]{Tb spectra after scaling; $\theta=2^\circ$
\label{beamK2}  beamK2.png  }
\end{figure} 
% how made: 

hemi flat.  11 207 252 


16 9=10 46 

theta-bar=       7.8983530
High slopes not covered=  0.000200987
@64 651 66
\ref{beamKp}
\begin{figure}[!ht] \igq{beamKp}
\caption[Two slopes]{Tb spectra for two roughnesses, $\theta = 10^\circ$ and 15\qd. KRC model set BeamK, four view directions as indicted in the legend.
\label{beamKp}  beamKp.png  }
\end{figure} 
% how made: beaming.pro, @113 123 66

 \qcite{Rozitis11} values in Table 2: DAU=1., SOLCON=1360, obliquity=0, Period=2551440/86400.=29.5306, Alb=1. EMIS=.9, INERTIA=50

For a circular pit, the slope distribution :
\qbn  p(\theta) =\frac{\sin x \ \cos x}{ \int_{x=0}^{\theta_\mathrm{max}} \sin x \ \cos x \ dx}  \qen

\qbn  p(\theta) \frac{\sin x \ \cos x} {\sin^2x_m/2} \qen

$\sin x$ comes from the circumference at angle $x$ and  $\cos x$ comes from the projection onto the horizontal plane.

\section{Testing sep 13}

Check Fig. \ref{beam5q}, 
\begin{figure}[!ht] \igq{beam5q}
\caption[Ts vrs azimuth]{Debug: BeamA 6200 lat=-6.04000 H= 6.50000 jh,jl= 6 8 .
  The reversal at low azimuth is where the Sun has not risen, but T's are higher
  due to late afternoon heating the prior day.
\label{beam5q}  beam5q.png }
\end{figure} 
% how made: 

\begin{verbatim}
 rotprt,sdrm,/help
9-vector Rotation matrix:  
   0.258819  -0.000000  -0.965926  0 3 6  x-- axis      x  y  z  axis
  -0.965926   0.000000  -0.258819  1 4 7  y-- of new    |  |  | of old
   0.000000   1.000000  -0.000000  2 5 8  z-- in old    |  |  | in new

 rotprt,sdct,/help
9-vector Rotation matrix:  
   0.258819  -0.965926   0.000000  0 3 6  x-- axis      x  y  z  axis
  -0.000000   0.000000   1.000000  1 4 7  y-- of new    |  |  | of old
  -0.965926  -0.258819  -0.000000  2 5 8  z-- in old    |  |  | in new


Expect      
 .2   -.9   0    
   0    0   1 
 -.9  -.2   0

\end{verbatim} 

%\clearpage %__________________________________________________________

\section{Testing sep 22}
 Indices in @64 loops i=hour, j=latitude, k=azimuth, m=slope, n=view.
 Use l=wavelength



theta-bar=       3.9793773 High slopes not covered= -1.19209e-07
theta-bar=       11.698571 High slopes not covered=    0.0216545
theta-bar=       18.716082 High slopes not covered=     0.236797

write  file: /work1/krc/beam/KLA5858H24L9w1.cub Size=  4 12 4 4 3 4 576

Effect of roughness for run K is shown in Figure \ref{KLA7}; the other runs are
similar, as shone in Figure \ref{KLAx}.

WARNING: My intuition is that the E/W profile would show more effect than N/S;
will check any literature examples.

With date 6090, near equinox, hour profile even weaker compared to N/S,
 so something must be wrong. must have incorrect rotation matrix.

\begin{figure}[!ht] \igq{KLA7}
\caption[Effect of roughness]{Effect of roughness on the flux spectrum for
  tilt-set K. Abcissa is wavelength (most rapid index) and view dierection, each
  is labeled near the bottom of the Figure; left side is profile in HOur and
  right side is profile in Latitude. Ordinate is flux normalized to tht at the
  center of the profile (the surface zenith, 6'th set from the left). The smooth
  model (zero roughness) is unity throughout, as expected. the effect of
  roughness increases toward short wavelength in all cases. Tilt-sets L and A
  are similar.
\label{KLA7} KLA7.png  }
\end{figure} 
% how made: beaming.pro 130, 7 

\begin{figure}[!ht] \igq{KLAx}
\caption[Flux profiles]{Profiles of flux excess for grid point Hour=12,
  Latitude=0. at date 5858. Left side is viewer traverse in hour, from 7 to 17,
  and is magnified by a factor of 10; the variation is about $\pm 4$\%.  Right
  side is viewer traverse in latitude from 75 S to 75 N. showing flux variation,
  normalized to viewing from the zenith, show almost 40\% variation at 5\um~ and
  about 7\% at 50\um. The left legend relates line colors to wavelengths, the
  right legend indicates the line type for each of the three tilt-set
  runs. Annotations near the figure bottom indicate the viewer direction;
  spacing in both profiles is 15\qd.
\label{KLAx}  KLAx.png  }
\end{figure} 
% how made:  beaming.pro 7 ,71

Check on Solar dec is in Fig \ref{slope20noon}, when the sdec plot @253 would indicate it was near minimum

\begin{figure}[!ht] \igq{slope20noon}
\caption[slope20noon]{Temperatures at all latitudes and azimuths at noon for
  slope 20\qd; using t4 for date=6090 and slope index 10. Sub-solar declination
  must be near the crossing point, as sun near zenith would minimize effect of
  azimuth.
\label{slope20noon}  slope20noon.png  }
\end{figure} 
% how made:

%\clearpage %__________________________________________________________

 \section{Testing Oct3+}
After fixing sun motion, erase all prior beam runs and do run for A alone, then
K and L at the same time; elapsed clock time about 26 min for one and 27 min for
two.

Make files thru radiance for 3 dates:
\qi 5900 Sun 33 N
\qi 6092 Sun 33 S
\qi 6100 Sun 0.6 N, spring equinox

 Run local views for several roughnesses. The fraction of high-angle slopes not
 covered by the KRC slope set is given in the following table; last 2 columns are
 with 40 and 60\qd~ max. slope files
\begin{tabbing}
inputWW \= not covered \=  High slopes\= High slopes \kill
$\theta$ \> $\overline{\theta}$ \> Max. slope \>  Max. slope \\
input   \> computed \> 40\qd  \> 60\qd \\
5 \> 3.98  \> -1.2e-07 \>  -1.2e-07 \\
15 \>  11.70 \>    0.0216 \>  0.0002 \\
25 \>  18.72  \>     0.2368  \>   0.0405 \\
35  \>  24.38 \>     \>  0.1584 \\
\end{tabbing}

 Wrote file /work1/krc/beam/BeamKLA5900H24L9w1.cub


Add source file to plot title. Result for sun 30N are in Figure 
\ref{30N72}
\begin{figure}[!ht] \igq{30N72}
\caption[Beam profiles with Sun 30N]{Viewer traverses at 30 N. See caption to Fig. \ref{KLAx}
\label{30N72} 30N72 .png  }
\end{figure} 
% how made: beaming 147 123 

 The Tb spectra for global views at H=12, L=0 for smooth models have Tb decrease
 from 5 to 50\um of 27.3K (run A) to 28.2K (run K), dotted lines in Figure
 \ref{Tb73KLA}.  Models with $\overline{\theta}=24.4$\qd have spectral
 differences over this wavelength range that are 3.34 $\pm$0.08 greater, as
 shown in Figure \ref{Tb73KLA}.

\begin{figure}[!ht] \igq{Tb73KLA}
\caption[Global Tb]{Spectral brightness temperature for 4 global views of a
  surface with $\overline{\theta}=24.4$\qd on day 5900. The highest albedo model
  (run A) has the lowest slope (weakest spectrum) and is about 4 K cooler than
  the Kheim model (run K) at 10 \um. The Lambert model (run L) is slightly
  warmer than the Kheim model in all views; about 0.5 at 5\um~ and about 0.9 at
  50\um. The dotted lines show the spectra for the smooth model for the 3 runs
  for the H=12 L=0 view.
\label{Tb73KLA} Tb73KLA .png  }
\end{figure} 
% how made: beaming 7,73

\end{document} %===============================================================

\ref{}
\begin{figure}[!ht] \igq{}
\caption[]{
\label{}  .png  }
\end{figure} 
% how made: 

\begin{table} \caption[]{}  \label{}
\begin{verbatim}
---
\end{verbatim}
\vspace{-3.0mm}
\hrulefill \end{table}  
