\documentclass{article}  
%\documentclass[draft]{article}  % See Skeleton.tex for examples of many things
% see definc.sty for other page format settings
%\usepackage{epsfig}
%\usepackage{definc}  % Hughs conventions  
\usepackage{hyperref}  % hyperlinks
\textheight=9.3in  \topmargin=-0.4in
\textwidth=7.0in  \oddsidemargin=0.0in \evensidemargin=-0.0in 
\parindent=0.em \parskip=1.ex %  no indent & paragraph spacing

\title{The KRC Planetary ORBit (PORB) System Users Guide}
\author{Hugh H. Kieffer  \ \ File=-/krc/PUG.tex 2014jun10}
% local definitions
%\newcommand{\short}{full}    % begin & end un-numbered equations
\newcommand{\qdp}{$. \! ^\circ \! $} % degree over decimal point NOT in math mode 

\newcommand{\ql}[1]{\label{eq:#1} \hspace{1cm} \mathrm{eq:#1} \end{equation}}
%\newcommand{\ql}[1]{\label{eq:#1} \end{equation} } % for final

%\newcommand{\cf}{$\Leftarrow$} % comes from

% had used  \nf \np \qi \qii \qiii \qcite
\newcommand{\np}{\textbf}  % name of program or routine
\newcommand{\nf}{\textit}  % name of file
\newcommand{\qcite}[1]{#1=\cite{#1}}  % show my .bib name
\newcommand{\qi}{\\ \hspace*{2.em}}      % indent 1
\newcommand{\qii}{\\ \hspace*{4.em}}     % indent 2
\newcommand{\qiii}{\\ \hspace*{6.em}}    % indent 3

\begin{document}
\maketitle
\tableofcontents
\hrulefill .\hrulefill
%\listoffigures
%\listoftables

This file is a guide to using PORB and is intended for the general user who
wants to apply KRC to someplace other than Mars.  A separate document
\nf{-/krc/porb.tex} describes the design of the PORB system; it should not be
needed by the general user.

\section{Introduction} 
This document describes Version 2.2.2 and later; it applies to KRC Version 2.2.1
and later; 2.2.2 includes additional source files \nf{small.tab} and
\nf{exoplan.tab}. The intention is to allow a user to create a \textbf{geometry
  matrix}, 7 ridgely formatted lines of numbers, for inclusion in a KRC input
file that would apply to any planet, satellite, asteroid, comet or exoplanet the
user may wish.

Version 2 of KRC uses dates relative to J2000.0 = 2000Jan01 noon UTC. To change
from version 1 dates, subtract 11545.0

\section{Outline}

Check \nf{PORBCM.mat} to see if it already contains the matrix you want. If so, simply copy that into the KRC input file. Otherwise proceed....

The executable of \np{porbmn} and the elements data files must be in the same directory. Copy them all into a directory of your choice.

Determine if the elements files to be used need any edits for a new object.
 If so, do the edits.

Run PORBMN to create a geometry matrix, see \S \ref{cook}; cut-and-paste it into your KRC input
file. Then run KRC.

\section{Files input to PORB} %------------------------------------------------
Files formatted for the PORB system cover four catagories, each file has its own format:
\qi Planets and satellites:  one file for orbits and another for spin-axes
\qi Minor (or small) bodies: orbits and pole direction for each object. Two versions.
\qi Comets: Orbit size specified by perihelion distance.  
\qi ExoPlanets: Orbit specified by semi-major axis, period and eccentricity; each commonly known. Relative orientation of the body spin axis specified by obliquity and Ls at periapsis, neither is commonly known.

All files contain adequate information in the header to easily add additional objects.

Reading of these files uses Fortran list-directed format, which uses white-space (blank or tab)
to separate items and will terminate with a slash '/'. Thus, names and numbers
must not have any internal space or /.  On each line, anything after the number
of items in the READ list is ignored, so comments may be entered. The first body in each file has a ``/'' just after the items read on each line to indicate how many items are read. All but the planets files contain a skeleton form for adding objects.

\subsection{Solar system Planets} 

\begin{description} 
 \item [\textit{standish.tab}] \ \qcite{Standish06} Keplerian elements and rates of change for 9 planets in mean ecliptic and equinox of J2000 valid for 1800 to 2050.
This file is valid for 1800 to 2050; it should not need any edits

Planet 10 is Mars derived from \qcite{Allison00} Table 2. 

 \item [\textit{spinaxis.tab}] \ \qcite{Archinal11} Planets: Direction of the pole and rates of change in ICRF, which differs from J2000 equatorial system by less than 0.1 arcsecond. The planets and a few satellites are included; more satellites can be added.
\end{description}
\subsection{Minor planets / asteroids }

There are two PORB-system ASCII files that differ slightly in format. Both
contain one element per line and include spin-axis.

\begin{description} 
 \item [\textit{minor.tab}] Generally older entries.

 \item [\textit{small.tab}]  Format designed to allow cut-and-paste from the JPL Small-Body Database 
Browser. Rotation period in hours. Must get spin axis from other sources.
\\ All entries were created 2013Aug 22 or later.
\qii includes comet Halley.
\begin{verbatim}
Look up object with JPL Small-Body Database Browser  
   http://ssd.jpl.nasa.gov/sbdb.cgi
Copy the number and name line. There must be no blanks or / within the name
    Append to this line the epoch
copy elements lines  e through M
Click on: Physical Parameters and copy 3 columns for the rotation period. E.g.,
  rot_per 	7.210 	h
Append 3 lines for: [data not in the JPL database]
0. / Right Ascension of Pole, J2000 [deg]   
0. / Declination of Pole, J2000 [deg]
0. / Prime meridian at epoch [deg]
   Then replace the zeros if you have real values.
\end{verbatim}

\end{description}

\subsubsection{Web sources} %.........................................

There are several sources on the web. Some I found did not have unambiguous definition of the coordinate systems used; these were established through email exchanges
\begin{description}  % labeled items   \item [] \item [] \end{description}

\item [PDS Small Bodies Node] \  http://pdssbn.astro.umd.edu/
\\ Primary source for spin axis.  Values are in ecliptic coordinates of equinox 1950
\vspace{-3.mm} 
\begin{verbatim}
 PDS Small Bodies Node http://pdssbn.astro.umd.edu/ 
DATA ARCHIVES> Archived at SBN > by target > Asteroids(all) 
  > Photometry and Lightcurves > EAR-A-5-DDR-DERIVED-LIGHTCURVE-V13.0 
Either download the set or:  > Browse > data >   lc_spinaxis.tab (230 kb)
      The explanation of this file is in  lc_spinaxis.lbl
Then scroll through. They are in numerical order with multiple entries per asteroid; 
the first line is 'summary' of the period (hours) and amplitude of the light curve.
 4th and 5th columns to the right of column containing 'summary' contains 
the 'best guess' of the longitude and latitude of the spin axis for each observer.
You get to choose your favorite! 
\end{verbatim}


\item [Minor Planet Center]  \  \url{http://www.minorplanetcenter.net/iau/info/OrbElsExplanation.html} 
\\ ArgPeri, Node and Inclin are in heliocentric ecliptic J2000.0
\qi Must specify date for elements that is within requested ephemeris range

\item [JPL Small-Body Database Browser]  \ http://ssd.jpl.nasa.gov/sbdb.cgi
\\ Orbit elements are in heliocentric ecliptic J2000.  Rotation period in hours. \ No spin-axis data.

\item [Ted Bowells database]   \ \url{ftp://ftp.lowell.edu/pub/elgb/astorb.html}
\\ Enormous; more than 1/2 million objects. Frequent updates. 
\\ ArgPeri, Node and Inclin are in J2000.0,  must be ecliptic. No spin-axis data.

\end{description}

\subsection{Comets}

\begin{description} 
 \item [\textit{comet.tab}] \  Some comets: orbit and spin. 
\\ WARNING: Consistency of source orientation systems uncertain
\qi Specify perihelion and eccentricity.
\\ Includes orbit and pole. Mostly compiled in 1980:1992
\end{description}
Current values are available at: 
\qi http://www.minorplanetcenter.net/iau/Ephemerides/Comets/Soft00Cmt.txt
\vspace{-6.mm}
\begin{verbatim}
   Columns   F77            Use
    1 -   4  i4     Periodic comet number
    5        a1     Orbit type (generally `C', `P' or `D')
    6 -  12  a7     Provisional designation (in packed form)
   15 -  18  i4     Year of perihelion passage
   20 -  21  i2     Month of perihelion passage
   23 -  29  f7.4   Day of perihelion passage (TT)
   31 -  39  f9.6   Perihelion distance (AU)
   42 -  49  f8.6   Orbital eccentricity
   52 -  59  f8.4   Argument of perihelion, J2000.0 (degrees)
   62 -  69  f8.4   Longitude of the ascending node, J2000.0
                      (degrees)
   72 -  79  f8.4   Inclination in degrees, J2000.0 (degrees)
   82 -  85  i4     Year of epoch for perturbed solutions
   86 -  87  i2     Month of epoch for perturbed solutions
   88 -  89  i2     Day of epoch for perturbed solutions
   92 -  95  f4.1   Absolute magnitude
   97 - 100  f5.1   Slope parameter
  103 -      a      Name
\end{verbatim}
J2000.0 was confirmed to be heliocentric ecliptic.

\subsection{exoPlanets} %--------------------------------- 

\begin{description} 
\item [\textit{exoplan.tab}] \ Major differences from solar system objects are:
\qi Need both semi-major axis and orbital period.
\qi Direction of spin axis is specified by planet obliquity and $L_S$ of periapses
\qiii Both these are rarely known, variations are illustrative.
\qi A skeleton for additions is at the bottom of the file.
\end{description}

\np{porbmn} run will print an estimated factor for the ``solar constant'' . This value times the normal SOLCON in KRC input should replace SOLCON. Currently the PORB system does not make use of the host star spectral type.

The length of the synodic day (assuming prograde motion) is also printed; it should be used as PERIOD in the KRC input. 


One list of stars with planets is: 
\url{http://en.wikipedia.org/wiki/List_of_exoplanetary_host_stars}
\subsection{Obsolete} %.....................

\textit{sturms.tab} \ \qcite{Sturms71} earliest source

\textit{seidelOld.tab} \ \qcite{Seidelmann74}  without Mars updates

\textit{seidel.tab} \ \qcite{Seidelmann74} Mars updated to 2008

\textit{MPcomets.tab} \ \qcite{} List of 251 objects from 
\qi http://www.minorplanetcenter.net/iau/Ephemerides/Comets/Soft00Cmt.txt
\qii Saved 2013 Jun 19 
% \pagebreak
\section{PORBEL: get elements} %--------------
Will read elements and spin-vector data for one object from a (or two) data file, move several values into common and return others.

For planets, reads first the orbital elements and adjusts to the requested epoch. 

Standish tables list the following with respect to the J2000 ecliptic system, in order, at J2000.0 and their derivative in time:
\qi 1 \ \ $a$: semi-major axis [au, au/century]
\qi 2 \ \ $e$: eccentricity [radians, radians/century]
\qi 3 \ \ $I$ : inclination [degrees, degrees/century]
\qi 4 \ \ $L$ : mean longitude [degrees, degrees/century]
\qii   $L=\Omega +\omega + M$  is a compound angle measured in two planes: 
\qiii From VE to node in plane of reference
\qiii then to periapsis and on to mean anomaly in orbital plane
\qi 5 \ \ $\varpi$ : longitude of perihelion [degrees, degrees/century] ($\varpi= \Omega+ \omega    $ ) 
\qii   $\varpi$ is a compound angle measured in two planes:
\qiii $\omega$, the argument of periapsis, measured in the plane of orbit 
\qi 6 \ \ $\Omega$:  longitude of the ascending node [degrees, degrees/century]
\qii  measured in the reference plane.

\bibliography{mars}   %>>>> bibliography data
\bibliographystyle{plain}   % alpha  abbrev 
\appendix

\section{Cookbook \label{cook}} %______________________________
\vspace{-3.mm}
\begin{verbatim}
        A Cookbook to running the  KRC Plantary ORBit system: PORB

cd into the directory containing the probmn executable and the *.tab files

> porbmn    To start the main program

Will be asked where to send output; normally enter:  T <CR>

Will be asked which of 5 options:
 Enter: 1

Will be asked which source data.
 Normally enter : 1 (planets) or 4 for asteroids or 6 for exoplanets

Will be asked which item in the file. Should have looked at the chosen elements
file to know this
 Enter the proper number.   E.g., 4 is Mars

Only if doing planets, will be asked for the epoch in centuries after 2000. 
E.g., for 2012,  enter 0.12

[ Version 3+ only  Will be asked which Debug print, enter 0 or /  ]
 
At this point, the calculations are done and a brief report will be appear
If and only if doing Planets, will be asked for which pole position
   a  /  will use the planet pole, else, enter a satellite name
      If that is not in the spinaxis table, will use the planets pole.

If doing exoPlanets, then note the solar power factor and the synodic day.

Will be asked which of 5 options:
  4  will print the values in vector format and come back to this same prompt
  2  Will begin the required output process: Then
   Will be asked to read or save.   Enter: 2
   Will be asked output format.   Enter: 2
   Will be asked for an output file name: 
     A single "/" with no leading spaces will use the default "PORBCM.mat" and
       append the latest set to whatever is already in that file.
     Else enter a new file name and <CR>

Will be asked which of same 5 options:
 Enter: 0 to quit  OR to another object.

You then copy the 7 lines in the output file into a KRC input file.
\end{verbatim}


\end{document} %===============================================================
% ===================== stuff beyond here ignored =============================
